\subsection{Les problèmes}
\subsubsection{Typage des Objets récupérés}
Il est très difficile de gérer la généricité en partant des résultats de la requête select. Il a fallut gérer les valeurs en tant qu'objet ce qui n'est pas forcément judicieux. Mais le fait de traiter les tables de manière dynamique oblige à procéder de cette façon. 

Il aurait été plus facile de traiter ce problème avec un langage dynamique. Dans ce cas là, il aurait été envisageable de créer à la volée les classes correspondantes aux tables.
\subsubsection{Construction des requêtes}
Le langage SQL n'est pas réputé pour sa simplicité d'utilisation. Les SGBD sont relationnels et le java est un langage objet, il y a donc toujours un certain nombre de difficultés pour la correspondance entre un tuple SQL et sa représentation en Java. La création des query à demandé un travail conséquent et les possibilités offertes par le programme sont limitées.
\subsubsection{Le traitement des mdb et xls}
La transcription du fichier mdb en script SQL n'est pas aisée. En effet, il n'est pas possible d'exporter la base et le format mdb n'est pas facilement exploitable. De plus, les champs de la base mdb et les entêtes du xls ne correspondent pas. Au final, la table créée correspond à une vue du xls.

\subsubsection{Servlets}
Un problème rencontré lors du développement avec des servlets/jsp a été les notions flous que nous avions concernant ces technologies. En particulier, on peut citer l'utilisation du ServletContext, l'utilisation d'un servlet comme contrôleur et d'une jsp comme vue, ou la façon de transmettre des données à la vue.

\subsubsection{Passage de paramètres ``typesafe''}
Lorsque l'on passe des paramètres d'une servlet à une jsp, on passe soit par la session soit par la requête. Or dans les deux cas, on récupère un objet de type Object à partir d'une chaîne de caractères. 
Ces deux points posent problème : si le développeur fait une faute de frappe ou si un autre développeur utilise le même nom comme ``clef'', retrouver la source du bug peut devenir cauchemardesque. 
Le transtypage obligatoire peut également poser problème : on suppose connaître à l'avance le type d'objet récupéré. Si ce n'est pas le cas, l'application récupèrera une exception, et aucun objet exploitable.
\subsubsection{Composants jsp}
Le dernier point problématique est l'absence de composants près à l'emploi, comme des formulaires, etc.
\subsection{Les limitations}
\subsubsection{Caractères non ASCII}
L'utilisation de lettres entre a et z ne pose ancun problème. L'utilisation de caractères accentués va poser problème avec l'encodage et la base de données. La requête SQL a alors de grandes chances d'échouer, comme avec l'utilisation de ' ou de \verb|\|.
\subsubsection{Requêtes sql limitées}
Les requêtes sql réalisables sont limitées par l'implémentation et l'interface que nous avons réalisées. Il n'est pas possible de faire des jointures entre les tables par exemple.
\subsubsection{La sécurité}
La sécurité de l'application laisse à désirer. Il n'y a aucune protection sur les requêtes sql et n'importe quel utilisateur peut faire une attaque par injection sql. 
Par exemple, il peut taper dans la clause WHERE.
\begin{verbatim}
 ';DROP TABLE APPAREL;
\end{verbatim}
Cela aura pour effet d'exécuter un DROP sur la table APPAREL.


