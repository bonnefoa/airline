\subsection{Typage des Objets recuperes}
Il est tres difficile de gerer la generecite en partant des resultats de la requete select. Il a fallut gerer les valeurs en tant qu'objet ce qui n'est pas forcement judicieux. Mais le fait de traiter les tables de manieres dynamiques oblige a proceder tel que cela. Il aurait ete plus facile de traiter ce cas avec un langage dynamique. Dans ce cas la, il aurait ete envisageable de creer a la volee les classes et les traitees ainsi.
\subsection{Construction des requetes}
Le langage SQL n'est pas repute pour sa simplicite d'utilisation. Les SGBD sont relationnels et le java est un langage objet, il y a donc toujours un certain nombre de difficultes pour la translation entre le SQL et le Java. La creation des query a demande un travail consequent et le possibilite offertes sont limitees.
\subsection{Le traitement des mdb et xls}
La transcription du fichier mdb en script SQL n'est pas aisee. En effet, il n'est pas possible d'exporter la base et le format n'est pas exploitable. De plus, les champs de la base mdb et les entetes du xls ne correspondent pas forcement. Au final, la table cree correspond a une vue du xls





@dd
not typesafe !
tout repose sur des string !
possibilite de collision !
notions flous pour les jsp  au depart (ServletContext ou l'url de base par ex)
pas de composant !